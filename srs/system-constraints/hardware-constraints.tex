\subsubsection{Computational Constraints}
  \label{system-constraints/hardware/computation-title}
  The University of Miami (UM) Center for Computation Science (CCS) will
  make available to this project its Pegasus computation platform. Pegasus
  allows for 220 teraflops of computational power and has over 3 petabytes of
  available storage.

  More information can be ascertained via
  \url{http://ccs.miami.edu/resources/compute-systems}.

  \begin{table}[h!]
    \caption{Computational Constraints}
    \label{system-constraints/hardware/computation-table}
    \begin{tabularx}{\textwidth}{|l|X|}
      \hline
      Title       & Computational Constraints. \\ \hline
      Description & Amount of computation and processing power available to
                    the application. \\ \hline
      Priority    & Low: 5. \\ \hline
    \end{tabularx}
  \end{table}

\subsubsection{Storage Constraints}
  As mentioned in Section \ref{system-constraints/hardware/computation-title},
  the Pegasus computation platform has over 3 Petabytes of available storage.
  Thus, this application will have very limited if not entirely nonexistent
  storage constraints. Indeed, the bulk of the storage needs will be occupied by
  the image data the application is to process, which has already been
  accomodated by the Pegasus system.

  See \url{http://ccs.miami.edu/resources/compute-systems} for more information.

  \begin{table}[h!]
    \caption{Storage Constraints}
    \label{system-constraints/hardware/storage-table}
    \begin{tabularx}{\textwidth}{|l|X|}
      \hline
      Title       & Storage Constraints. \\ \hline
      Description & Amount of memory and storage available to the application.
                    \\ \hline
      Priority    & Low: 5. \\ \hline
    \end{tabularx}
  \end{table}
